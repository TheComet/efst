\documentclass[notitlepage]{fhnwreport}

\usepackage[T1]{fontenc}
\usepackage[english]{babel}
\usepackage{csquotes}
\usepackage{lmodern}
\usepackage{kpfonts}
\usepackage{hyperref}
\usepackage[backend=bibtex,bibstyle=numeric,sorting=nyt]{biblatex}
\usepackage[para,online,flushleft]{threeparttable}
\usepackage{booktabs}
\usepackage{lipsum}

\addbibresource{bibliography.bib}

\title{Memento -- Never answer the phone}
\author{Alex Murray}

\begin{document}
\maketitle

In this report, we  will  be  analyzing  some  of the camera work, editing and
lighting of a short scene  from  the  movie \textit{Memento}. The scene can be
found on youtube\cite{ref:scene}.

\subsection*{Camera Work}

Table \ref{tab:camerawork} breaks down all of the shots.

\begin{center}
\begin{threeparttable}
    \caption{Camera Work}
    \label{tab:camerawork}
    \begin{tabular}{p{3cm}p{11.2cm}}
        \toprule
        \textbf{Time} & \textbf{Description} \\
        \midrule
        \textbf{(1)} 0:00-0:03 & Close-Up shot of arm, slowly zooming in as he starts to peel away at his bandage. \\
        \textbf{(2)} 0:03-0:06 & Medium shot of his face, as he talks on the phone. The camera is also slowly zooming in. \\
        \textbf{(3)} 0:06-0:10 & Closer-Up shot of arm, again still zooming in has he continues peeling. \\
        \textbf{(4)} 0:10-0:14 & Medium/close shot of his face, he continues talking on the phone. \\
        \textbf{(5)} 0:14-0:15 & Macro shot of arm, he finally peels away the bandage but we don't get to see what's under it yet. \\
        \textbf{(6)} 0:15-0:17 & Close-Up shot of his face, he stops dead in his sentence as he looks at what is under the bandage. \\
        \textbf{(7)} 0:17-0:19 & Macro shot of what was under the bandage; it's some text tattoo'd on, saying ``Never answer the phone'' \\
        \textbf{(8)} 0:19-0:29 & Close-Up shot of his face as he asks, confused, ``Who is this?'' before the caller hangs up. The camera is focused on the center of his face and pans left and right when his head turns left or right. \\
        \textbf{(9)} 0:29-0:33 & Wide shot of the room with him sitting on his bed in the center. \\
        \bottomrule
    \end{tabular}
\end{threeparttable}
\end{center}

\subsection*{Editing}

I believe this scene was filmed using  only  three  shots, and then chopped up
during editing to make  it more captivating. Shots \textbf{(1)}, \textbf{(3)},
\textbf{(5)} and  \textbf{(7)}  possibly  all  belong  to a single, continuous
shot, and the same is true  for shots \textbf{(2)}, \textbf{(4)}, \textbf{(6)}
and \textbf{(8)}.

It's  subtle, but the camera slowly zooms in during both arm and  face  shots,
further contributing to the tension of the scene as  the  viewer  wonders more
and more what's under the bandage.

They switch away from shot  \textbf{(7)}  very  fast,  hiding what's under the
bandage  further still. Again, this is meant to build up tension until finally
the text is revealed in shot \textbf{(8)}.

\subsection*{Lighting}

In the first two shots, the colours have a very high  contrast.  It seems like
there's a bright light illuminating the arm  from  the back and a bright light
is illuminating  only  one  side  of  his  face,  leaving the other side dark.

This  is  because  he's  in a room with the lights off, and the only source of
light is that sun  shining  through the glass doors that (probably?) lead to a
balcony.

\subsection*{Sound}

I didn't realise at  first, but the volume of his voice has the same volume as
the crinkling sound of the bandage.  The  sound of the bandage is very raw and
close, as if a microphone is just millimeters away, even when  the  camera  is
focused on his face.

When  he  reads what's tattoo'd onto his arm, a weird distorted  high  pitched
bell  mixed  with  a  deep  synth fades in, to help emphasize the shock he  is
experiencing. Mixing high tones with low tones (and leaving out all mid tones)
is a common technique in  sound design to make the viewer feel uneasy, because
it  conveys  mixed messages. We are biologically programmed to find comfort in
low frequency sounds, whereas high  frequency  sounds  are  generally  trigger
alertness. Mixing the two makes us confused and uneasy.

\printbibliography

\end{document}
