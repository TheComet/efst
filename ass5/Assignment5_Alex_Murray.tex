\documentclass{fhnwreport}

\usepackage[T1]{fontenc}
\usepackage[english]{babel}
\usepackage{csquotes}
\usepackage{lmodern}
\usepackage{kpfonts}
\usepackage{hyperref}
\usepackage[backend=bibtex,bibstyle=numeric,sorting=nyt]{biblatex}
\usepackage[para,online,flushleft]{threeparttable}
\usepackage{booktabs}
\usepackage{lipsum}
\usepackage{float}
\usepackage{multicol}
\usepackage{titling}
% Post title
\newcommand{\subtitle}[1]{%
  \posttitle{%
    \par\end{center}
    \begin{center}\large#1\end{center}
    \vskip0.5em}%
}

\addbibresource{bibliography.bib}

\title{Treasure Planet}
\subtitle{Critical Analysis}
\author{Alex Murray}

\begin{document}

\maketitle
\tableofcontents
\newpage

\begin{multicols}{2}

\section{Introduction}

\textit{Treasure  Planet}   represents  the  peak  in  traditional  hand-drawn
animation featuring some of the most advanced techniques to  date  and heavily
utilised  Disney's \textit{Deep Canvas} technology to emulate moving shots  in
3D space.

It  also  represents  the  nail  in  the  coffin  for  traditional  hand-drawn
animation, as it was  single  handedly  responsible for shutting down Disney's
``Happiest Studio'' due to its massive failure.

There is  much  controversy  surrounding  the  circumstances  of  the  movie's
failure, because as  this  critique will reveal, objectively, \textit{Treasure
Planet} has everything required to be a successful movie.

\section{Background}

\textit{Treasure Planet} is a reimagination of the series \textit{Treasure Island}

\section{Plot}

\section{Camera Work, Sound, and Editing}

\section{Mise en Sc\'ene}

\section{Theme}

\section{Conclusion}

\printbibliography

\end{multicols}

\end{document}

