\documentclass[notitlepage]{fhnwreport}

\usepackage[T1]{fontenc}
\usepackage[english]{babel}
\usepackage{csquotes}
\usepackage{lmodern}
\usepackage{kpfonts}
\usepackage{hyperref}
\usepackage[backend=bibtex,bibstyle=numeric,sorting=nyt]{biblatex}
\usepackage[para,online,flushleft]{threeparttable}
\usepackage{booktabs}
\usepackage{lipsum}

\addbibresource{bibliography.bib}

\title{The Deer Hunter : One Last Shot}
\author{Alex Murray}

\begin{document}
\maketitle

In this report,  we  will  be  looking  at the second game of Russian Roulette
between Michael  Vronsky  (played by Robert De Niro) and Nikanor Chevotarevich
(played by Christopher Walken) in the  movie  \textit{The  Deer  Hunter}.  The
scene can be found on youtube\cite{ref:scene}.

\begin{center}
\begin{threeparttable}
    \begin{tabular}{p{4.2cm}p{11cm}}
        \toprule
        \textbf{Aspect} & \textbf{Description} \\

        \midrule
        Genre & 

War / Drama \\

        \midrule
        Plot and themes &

Nick and Michael were captives in the Vietnam war and were  previously  forced
to play Russian Roulette against each other by their captivators for some sick
entertainment.  They miracilously escaped by asking for 3 bullets instead of 1
to  make the game more interesting, and then shooting the captivators  at  the
right  moment  in  time.  This  scene is said to be one of the best  in  movie
history and you should definitely  go  watch  it  too!  Nick  recuperates in a
military  hospital  in  Saigon  with no knowledge of his friends. After  being
released, he goes AWOL and aimlessly  stumbles  through the red-light district
at night. Lost,  he  starts  playing  Russian  Roulette  for  money. In a last
attempt to save his friend,  Michael  joins  the game of Roulette and tries to
jog Nick's memory. But Nick knows he can't go back to the life he  had  before
the war, and decides to test his luck one final time. \\

        \midrule
        Setting (inside/outside) &

The scene takes place in a gambling den in the Red-light district  in  Saigon,
Vietnam. \\

        \midrule
        Use of camera (camera shots/movement)  &

The camera is fairly typical. It alternates between two  frontal views of Nick
and Michael, a  shoulder-shot  from  behind  Michael and an overall view where
Nick, Michael and the game's moderator are visible. \\

        \midrule
        Lighting and colour &

There is a fairly stark white light illuminating them from above. There are no
other lights in the room as far as I can tell. \\

        \midrule
        Sound and music &

There is no music. The sound alternates between being very  silent  (so we can
focus on the sound of  the  chamber  clicking  and the two men talking to each
other) and being very loud, with  people arguing in the background in betweeen
turns.\\

        \midrule
        Props &

The only prop being used in this scene is the revolver. \\

        \midrule
    \end{tabular}
\end{threeparttable}
\end{center}
\begin{center}
\begin{threeparttable}
    \begin{tabular}{p{4.2cm}p{11cm}}
        \midrule
        Costume &

Nick looks like a broken man. He has bags under  his  eyes, his face looks old
and used, yet he's wearing a  clean, white collared shirt and a red headscarf.
This  could possibly be a way to symbolise his heroic ability to  survive  the
many games of Roulette that he has. Michael  is  wearing  a  green  suite  and
slightly  lighter  green  shirt  underneath.  He  looks clean and well groomed
overall. Actually, everyone in  the room, even the crowed, is wearing collared
shirts. \\

        \midrule
        Acting/performance &

Christopher Walken does an amazing  job  at  expressing the mixed emotions the
character feels before pulling the second trigger,  and  of  course,  no other
actor but him can  pull  off  the iconic deranged stare required for the first
half of the scene. \\

        \midrule
        Spoken language & 

There's not much to say about the spoken language. \\

        \midrule
        Facial expression and body language &

The first time Nick pulls  the  trigger,  his  face  is cold and detached from
reality. This is a man with nothing to lose  and  everything  to  gain. Before
Michael picks up the  gun,  there  is a shot of Nick head-on who looks down at
the gun  for a moment and then back at his opponent, as if challenging him. As
Michael picks up the gun, holds it  to  his  head and says: ``I love you man''
before pulling the  trigger  slowly,  we  see  Nick  begin  to break a little.
Michael clearly doesn't want to be here -- he has a family and a life  to tend
to  --  but he's here and he pulled the trigger because his friend Nick  means
everything to him. At first Nick almost  looks  a  little  astonished (he nods
slightly, acknowledging the situation) and  transitions  into  disbelief. Nick
slightly shakes his head and squints his eyes  (it's  very  subtle), wondering
why, of all people in the world, Michael has decided to  come and play against
him. Michael asks him to come home,  at  which  point  Nick's face transitions
into sorrow. He wants to come home, more than anything, but he cannot. He will
never be able to return to a normal life after what has  happened.  Words fail
him. He lifts the gun but Michael stops him and asks him  to  remember the old
days. The trees, the  deer  hunting.  Nick  remembers  and utters: ``Yeah, one
shot.'' before lifting  the  gun  to  his  head  and pulling the trigger, thus
killing himself. \\

        \bottomrule
    \end{tabular}
\end{threeparttable}
\end{center}

``One shot'' is in reference to what they talked about when they used  to hunt
deer, and  has  multiple  meanings  in  this context. For one, Nick is showing
Michael that he is not in some drug crazed fog and remembers everything. ``One
shot'' is also the whole point of the  movie  and  indeed  of life itself: You
have one shot at  life  and  you have to do it right. Nick feels he screwed up
and thus, decides to end his life in ``One shot.''

By analysing this scene as an ``active viewer'' it has made me think about why
the characters acted as they did more in depth.

\printbibliography

\end{document}
